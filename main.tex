\documentclass[a4paper,12pt]{scrartcl}
\usepackage[margin=3cm,right=6cm]{geometry}
\usepackage[utf8]{inputenc}
\usepackage[T1]{fontenc}
\usepackage[osf]{libertine}
\usepackage[pdfencoding=auto]{hyperref}
\usepackage{bookmark} % faster updated bookmarks

\newcommand*{\abbr}[1]{%
   % Format abbreviations as small capitals
   \texorpdfstring{\textsc{\MakeLowercase{#1}}}{#1}%
}

\title{Managed Infrastructure with Automation}
\author{Tom Marcoen}
\begin{document}
\maketitle


%=======================================================================
\section{Introduction}
\label{sec:introduction}
The \abbr{MIA} project is based around monitoring.
Everything starts with setting up the necessary monitoring and then implementing the required changes to make the monitoring check succeed.
It's comparibly to test-driven programming where you first write the check and then the function to satisfy the check.
This way the monitoring is never an afterthought and you know from the get-go that the monitoring works.


%=======================================================================
\section{Scope of the project}
\label{sec:scope}
The main focus of the project will be web and mail hosting, including \abbr{DNS}.
We will assume a greenfield deployment in each of these three scenarios though this will be challenging for the setup of \abbr{DNS}.

%-----------------------------------------------------------------------
\subsection{In scope}

\paragraph{Mail services}
Mail services will include setting up of all necessary \abbr{DNS} records, a Postfix email server for a single domain, an \abbr{IMAP} server for mail retrieval and possibly a webmail client though this will require setting up a website, load balancer\ldots

\paragraph{Web services}
Setting up a website will include the application server, database, web server, load balancer, certificates, and \abbr{DNS} records.

\paragraph{Single sign-on}
Authenticating against the different services and especially for accessing your emails should be possible using your Microsoft Active Directoy account.

\paragraph{Remote access \abbr{VPN}}
This topic might become in-scope if I can find a way to make it work easily.
The problem will be configuring the local firewall to allow for inbound traffic.

\paragraph{Virtualization}
There will be a choice between spinning up new virtual machines, creating containers, or jails (FreeBSD).
In case a new virtual machine (cloud \abbr{VPS}) is created for each server, costs may increase significantly but the general performance might be better.
Linux containers and FreeBSD jails will require an initial set of cloud \abbr{VPS} instances to run the containers or jails on.

%-----------------------------------------------------------------------
\subsection{Out of scope}
The local network will out of scope, i.e., this project does not aim to monitor and configure local switches or firewalls.


%=======================================================================
\section{Initialization}
\label{sec:initialization}
During the initial setup the decision will have to be made for virtual machines or jails.
Either way at least two new virtual machines will have to be created with Digital Ocean or AWS.
Perhaps a manual option should be provided so that you can enter the \abbr{IP} addresses or domain names and the root credentials of two VMware vSphere virtual machines hosted locally.

Next you need to enter your domain name.
If you want to import the existing (public) zone configuration, you will either have to upload the zone file or allow the \abbr{MIA} box to do an \abbr{AXFR} zone transfer from the existing \abbr{DNS} servers.
Either way, manual actions will be needed to setup your two new name servers as the authoritative name servers for the domain.
Example configurations should be provided for a few major hosting companies, e.g., GoDaddy.
Even better would be if they provide an \abbr{API} so \abbr{MIA} can do these actions automatically.


%=======================================================================
\section{Single sign on}
\label{sec:sso}
The first two servers have been setup to host the (public) domain name.
Now we should get \abbr{SSO} to work so you can easily connect to these and future servers and services.





\end{document}