\documentclass[a4paper,12pt]{scrartcl}
\usepackage[margin=3cm,right=6cm]{geometry}
\usepackage[utf8]{inputenc}
\usepackage[T1]{fontenc}
\usepackage[osf]{libertine}
\usepackage[pdfencoding=auto]{hyperref}
\usepackage{bookmark} % faster updated bookmarks

\newcommand*{\abbr}[1]{%
   % Format abbreviations as small capitals
   \texorpdfstring{\textsc{\MakeLowercase{#1}}}{#1}%
}

\title{Managed Infrastructure with Automation}
\author{Tom Marcoen}
\begin{document}
\maketitle



\section{Introduction}
The goal of the \abbr{MIA} project is to setup a complete network infrastructure based on monitoring.
Let's take a website as example.
The website begins with the \abbr{DNS} name so we build this into the monitoring environment: an \abbr{A} record check for the website we want to build.
Apart from the \abbr{A} record the monitoring environment will also check for \abbr{NS} records and monitor those name servers.

If a check fails we try to automatically rectify the problem.
In case of a missing \abbr{A} record, we will create the record with Ansible.
In case the name servers are down, we will spin up new servers on Digital Ocean or AWS with Terraform, install the needed software, modify \abbr{NS} and \abbr{A} records to point to these new servers and pusht he \abbr{DNS} configuration using Ansible.

Next we will poll the \abbr{DNS} name for ports \abbr{TCP}/80 and \abbr{TCP}/443.
The same rules apply here.
If they are down, spin up servers with Terraform, install the software, push the configuration files and get the web server running.
Better yet, get HAProxy running and spin up web servers as the backend behind HAProxy.
Monitor the load of your web servers and if the load becomes too high, spin up additional servers.



\section{Options}
Instead of spinning up new virtual machines in the cloud using Terraform, there should be an option to spin up container/jail instances on a given set of virtualization hosts.
This will obviously reduce the costs on the number of virtual machines, but will increase complexity as those virtualization hosts will need additional monitoring to determine whether extra containers/jails are permitted to be installed.



\section{Configuration files}
The end user will have to define configuration templates which can be pushed using Ansible.
This means that the user will have to provide at least a few options from which a Apache configuration file will be generated.
The same goes for \abbr{DNS}, HAproxy et cetera.
Though it is not likely the end user will have to modify anything to the default two-node name server setup and most load balancer configurations do not need any tweaking either.



\section{Monitoring}
The heart and soul of \abbr{MIA} is of course the monitoring dashboard from which you can see the health of the environment.
Not only does it have to provide a generic overview of the customer service (e.g., the website) but it will also have to account for dependancies.



\section{Services which are in scope}
\begin{description}
\item[Website]
    Set up \abbr{DNS}, Let's Encrypt, load balancer, web server, application server (\abbr{PHP} or Python)\ldots
\item[Email]
    Set up \abbr{DNS} records, webmail client, mail server, possibly also an IMAP server\ldots
\item[\abbr{DNS}]
    Implement \abbr{DNSSEc}
\item[\abbr{VPN}]
    Remote access \abbr{VPN} based on Wireguard or something similar.
\item[Single sign-on]
    Provide \abbr{SSO} with Microsoft \abbr{AD} integration.
\end{description}




\section{Services which are out of scope}
Currently there are no plans to include monitoring and configuration of the local network.
This is in part because there are too many switches and firewall in use by small businesses of which many have no \abbr{CLI} to manage them or provide any metrics (e.g.,\ \abbr{SNMP}).



\section{Workflow for a website}
When you want to add a new website, you start with entering the \abbr{FQDN}, e.g., www.marcoen.net.
This will immediately setup several checks:
\begin{itemize}
\item
    Perform a \abbr{DNS} lookup for the given \abbr{FQDN}.
\item
    Ping the \abbr{IP} address returned by the \abbr{DNS} lookup.
\item
    Check whether both \abbr{TCP}/80 and \abbr{TCP}/443 are opened on the given \abbr{IP} address.
\item
    Check whether the returned web page matches the expected web page.
\end{itemize}
At first all these checks will fail.
Some failed checks will trigger an action, e.g., when the \abbr{DNS} lookup fails, try to add the \abbr{DNS} record.
This can quickly lead you down a rabit hole though.
You will need to ask the user for credentials to authenticate to the master name server, or set it up if it does not yet exist.
You will need to ask the user for the \abbr{IP} address---something the user should not need to know.

\end{document}